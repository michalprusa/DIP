%% This is an example first chapter.  You should put chapter/appendix that you
%% write into a separate file, and add a line \include{yourfilename} to
%% main.tex, where `yourfilename.tex' is the name of the chapter/appendix file.
%% You can process specific files by typing their names in at the 
%% \files=
%% prompt when you run the file main.tex through LaTeX.
\chapter{Úvod}
Téma anténních struktur osazených čočkami, zejména anténních čoček jako takových, bylo vemi aktuální a rozvíjené v počátcích vývoje antén pro mikrovlnnou techniku. Avšak s příchodem reflektorových antén se velká část pozornosti odklonila právě k nim zejména z důvodu jejich vyšší efektivity. Poslední dobou se stále se zvyšujícím kmitočtem, anténní struktury s čočkami začínají opět získávat svoji pozornost. \cite{ModernLens}

\section{Motivace}
3D tisk je technologie zejména pro výrobu rychlých prototypů, takzvaný "rapid prototyping", používaná ve stále více oborech. S uvedením speciálních polymerních materiálů vykazující vyšší elektrickou vodivost do prodeje má stále větší smysl využití právě této technologie pro urychlení vývoje a malosériovou výrobu antén (s výjímkou dielektrických rezonančních struktur).

\section{Cíl}
Primárním cílem této práce je výzkum využití 3D tiskové technologie FDM pro výrobu anténní struktury (trychtýřová anténa s dielektrickou čočkou) od návrhu optimalizovaného pro jednoduchou výrobu, přez extrakci dielektrických parametrů po průchodu technologickým procesem, po vlastní realizaci navržené struktury na frekvenci 10\,GHz. Sekundárním cílem práce byl průzkum možností následného pokovení pro minimalizaci rozdílu mezi "vytisknutou" a profesionálně realizovanou strukturou.