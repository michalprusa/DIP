% $Log: abstract.tex,v $
% Revision 1.1  93/05/14  14:56:25  starflt
% Initial revision
% 
% Revision 1.1  90/05/04  10:41:01  lwvanels
% Initial revision
% 
%
%% The text of your abstract and nothing else (other than comments) goes here.
%% It will be single-spaced and the rest of the text that is supposed to go on
%% the abstract page will be generated by the abstractpage environment.  This
%% file should be \input (not \include 'd) from cover.tex.

\section*{Anotace}

Obsahem této diplomové práce je výhradně rozbor využití FDM/FFF technologie 3D tisku ve vysokofrekvenční technice, konkrétně možnosti realizace trychtýřové antény s dielektrickou čočkou pro optimalizaci vyzařovacích vlastností. V první části se práce zabývá trychtýřovými anténami a jejich návrhem, následně stejným postupem dielektrickými čočkami. Dále se práce zabývá materiály pro 3D tisk, jejich parametry, včetně extrakce a popisu metody. Závěrem práce je popis realizace navržené antény, předvedeny výsledky a porovnány se simulací.
Postupy popsanými v této práci se podařilo realizovat funkční trychtýřovou anténu s dielektrickou čočkou pomocí 3D tisku, bohužel s velmi nízkým ziskem, a extrahovat parametry běžných materiálů po průchodu procesem.


\section*{Klíčová slova}

3D tisk, RepRap, Trychtýřová anténa, Anténní čočka, Dielektrická čočka, Extrakce parametrů

\section*{Abstract}

Content of this masters thesis is specially a research of possible usage of FDM/FFF 3D printing technology in high frequency technology, specifically realization of horn antenna with dielectric lens for optimization of radiation properties. In the first part, the thesis is explaining horn antennas and it's design, then dielectric lenses in similar way. Then the materials for 3D printing is discussed, described properties and it's extraction, including description of the method. At the end, realization of designed antenna and lens is described, presented results and compared to simulation.
With methods described in this thesis, we were able to realize working horn antenna with dielectric lens using 3D printing technology, unfortunately with very low gain, and extract parameters of common materials after printing process.


\section*{Key words}

3D printer, RepRap, Horn antenna, Antenna lens, Dielectric lens, Parameter Extraction