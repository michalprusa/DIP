%% This is an example first chapter.  You should put chapter/appendix that you
%% write into a separate file, and add a line \include{yourfilename} to
%% main.tex, where `yourfilename.tex' is the name of the chapter/appendix file.
%% You can process specific files by typing their names in at the 
%% \files=
%% prompt when you run the file main.tex through LaTeX.
\chapter{Závěr}
Cílem této diplomové práce bylo realizovat trychtýřovou anténu s dielektrickou čočkou technologií 3D tisku s hlavním zaměřením na optimalizace struktur vyhovujících právě použité technologii. 

Struktury byly úspěšně realizovány jak z vodivých, tak standartních termoplastických polymerů, včetně čočky. Funkčnost vytvořených struktur byla ověřena měřením.

Na základě změřených směrových charakteristik (obrázek \ref{fig:FarfieldsE}, \ref{fig:FarfieldsH}) a jejich výsledků nebylo relevantní pokračovat v měření zisku realizovaných struktur vlivem velmi malého přijatého výkonu, což je způsobeno velkými ztrátami na strukturách (jak F-electric "vodivého" filamentu, tak při následném pokovení EMI 35).

Pro zlepšení parametrů, bylo předvedeno i řešení s možným následným pokovením, kde byla projevena velká iniciativa ze strany společnosti Electroforming s.r.o. na základě předvedených vzorků realizovaných elektricky vodivým materiálem, zejména z důvodu možného vypuštění nebezpečných procesů, které je nutno podstupovat při použití standartních polymerních materiálů. Bohužel se však toto řešení nepodařilo zdárně realizovat z důvodu vysokých teplot při procesu což způsobilo deformaci struktury, viz obrázek \ref{fig:PLdef}. Galvanizační lázeň lze ale optimalizovat pro provoz při nižších teplotách.

Dále se naskýtají možnosti využití v tisku ryze dielektrických struktur, jako například rezonátorů. Je tedy velká motivace rozvíjet dále možnosti aplikace této technologie, jelikož má velký potenciál pro tisk materiálů s vnitřními strukturami oproti ostatním technologiím, dále je třeba nezapomenout že vodivé materiály, které byly v této práci použity, jsou jedny z prvních dostupných a stále probíhá vývoj s cílem přiblížení se ideálním parametrům.

